% vim: tw=80 noai
\documentclass[a4]{article}
%\usepackage[latin1]{inputenc}
\usepackage[utf8]{inputenc}
%\usepackage[T1]{fontenc}
\usepackage[brazil]{babel}
\usepackage{url}
\usepackage{graphicx}
\usepackage{listings}
\usepackage{verbatim}
\usepackage{subfigure}
\usepackage{multicol}
\usepackage{framed}

%\hyphenation{ab-stract}

%%%%%%%%%%%%%%%%%%%%%%%%%%%%%%%%%%%%%%%%%%%%%%%%%%%%
\def\lstlistingname{Listagem}

% Comandos
%\newcommand{\TODO}{\textbf{TODO:} }

\newcommand{\web}{\emph{web}}

\begin{document}

\begin{abstract}
% Kent Beck's four sentences for a good abstract:
%% the problem
%% why the problem is a problem
%% startling sentence
%% implication of the startling sentence

  Muitos sistemas não têm uma arquitetura bem documentada.
  Sem arquitetura bem documentada é difícil entender o sistema e planejar mudanças
  O uso de tecnicas de mineracao e algoritmos de clustering ajuda a construir uma visão arquitetural do sistema.
    Avaliamos algumas dessas técnicas
  Essas tecnicas revelam bla-bla-bla.
  

bli~\cite{Maqbool2007}

\textbf{Keywords:} 
aspect-oriented modeling,
reverse engineering,
software visualization.
\end{abstract}

% Introducao sobre recuperacao arquitetural (e trabalhos relacionados). 
% Clustering
%% Descricao das entidades, descricao de acoplamento, ...
%% Caracteristicas de software
% Algoritmo hierárquico aglomerativo e técnicas associadas
%% AHA - merges são finais
%% AHA - algoritmo guloso, otimização local
%% AHA - medidas monotonicamente relacionadas (
%% AHA - usa matriz de similaridade, que pode codificar qualquer critério
% Experimento
%% Sistemas avaliados
%%% Caracteristicas: densidade, similaridade media, desvio-padrao da simil.
%% Criterios de avaliacao (authoritative..., size, stability)
% Resultados experimentais e sua interpretacao
%% Threats to validity.
%%% As métricas são uma merda. O extractor funciona bem?
%% Avaliacao quantitativa
%% Avaliacao qualitativa
% Discussao
%% O que significa esse particionamento? Pra que serve?

%%
\section{Introdução} \label{intro}

O conjunto de decisões mais significativas de um software formam 
arquitetura do software. Essas decisões devem ser tomadas cedo no processo
de desenvolvimento do software, pois elas afetam um grande número de decisões
que devem ser tomadas ao longo do desenvolvimento.

As decisões arquiteturais são motivadas pelos requisitos do sistema e dizem
respeito a uma parte de seus \emph{stakeholders}. É comum dividir a
arquitetura em diversas visões, cada uma delas contemplando um subconjunto
dos requisitos e dos \emph{stakeholders}. A visão lógica, por exemplo,
é destinada principalmente aos \emph{designers} e programadores e trata de
requisitos como a facilidade de manutenção e de compreensão do \emph{software}.
[cite 4+1]

O campo de pesquisa denominado recuperação de arquitetura de software tem como
objetivo extrair aspectos da arquitetura de um sistema a partir de artefatos
criados durante seu desenvolvimento. A maioria dos trabalhos na
área se concentra na recuperação de uma visão lógica da arquitetura de um
sistema a partir da análise de seu código-fonte.

Uma boa porção desses trabalhos emprega algoritmos de agrupamento
(\emph{clustering}) para identificar grupos de entidades de código-fonte
(tipicamente funções e classes). Um bom agrupamento deve
fornecer a desenvolvedores uma visão geral do sistema e deve revelar atributos
de qualidade do software como reusabilidade, manutenibilidade e facilidade de
compreensão.

%% trabalhos relacionados?? Anquetil e Lethbridge, Maqbool e Babri.



%%%%
\section{Recuperação arquitetural a partir do código-fonte}

A recuperação da arquitetura de um sistema a partir de um código-fonte pode
ser dividida em duas etapas: a extração do design do sistema e a abstração
desse design para uma descrição arquitetural.

Para extrair o design, usamos a ferramenta Design Wizard, que faz análise
estática de programas escritos em Java. O design é 
descrito por um grafo onde os vértices representam as classes do sistema
e as arestas representam dependências entre as classes. Essa dependência
ocorre em diversas situações --- por exemplo, quando uma classe possui uma
referência para um objeto de outra classe, ou quando um de seus métodos chama
algum método de outra classe.

A etapa de abstração consiste na aplicação de algoritmos de agrupamento sobre
as classes do design. O resultado é um particionamento do sistema en módulos
arquiteturais concebidos de forma que cada módulo contém classes similares.

É necessário, portanto, quantificar a similaridade entre cada par de classes
de um sistema. A métrica de similaridade pode considerar diversas 
características das classes, tais como identificadores e comentários a elas
associados, ou mesmo seu histórico de modificações.
%os nomes das classes e comentários a elas associados, a identidade dos
%desenvolvedores que modificaram a classe ou a quantidade de modificações.
Neste trabalho, no entanto, a métrica de similaridade é definida a partir
das relações de dependência entre as classes.

  [Figura: matriz de adjacências, representação de uma classe, grafo
  correspondente]

Existe uma literatura vasta sobre algoritmos de agrupamento. Neste trabalho
avaliaremos o algoritmo de agrupamento hierárquico aglomerativo.

\subsection{Algoritmo hierárquico algomerativo}

O algoritmo hierárquico aglomerativo (HA) produz, a partir de um conjunto
de classes e de uma métrica de similaridade entre classes, uma hierarquia
de grupos (\emph{clusters}), como mostra a figura \ref{fig:dendograma}. 
Cada grupo da hierarquia contém dois grupos menores.
%Nessa hierarquia, dois módulos podem ser agrupados em um
%módulo maior.
%Nessa hierarquia, cada módulo contém outros dois módulos, com
%exceção dos módulos unitários, que contém uma classe.
A hierarquia pode ser
representada por uma árvore (chamada de dendograma) cujas folhas
representam grupos unitários (que contêm apenas uma classe) e a raiz
representa um grupo que contém, transitivamente, todas as classes do
sistema.

  [Figura do livro - dendograma]

Chamamos de grupo raiz um grupo que não está contido em nenhum outro
grupo.
O algoritmo considera inicialmente cada classe como um grupo unitário e, a 
cada passo, agrupa os dois grupos raízes mais similares 
até que reste apenas um grupo raiz.

  [Algoritmo]

A similaridade entre dois grupos pode ser definida de diversas maneiras.
No Single Linkage, a similaridade entre dois grupos é a similaridade
entre os elementos mais similares --- um elemento de cada grupo.
No Complete Linkage, esse valor é a similaridade entre
os elementos menos similares --- um elemento de cada grupo.
Group Average...

  [Figura do livro - linkage]

Para obter um particionamento do sistema sob análise em módulos, basta
cortar o dendograma em algum ponto. 
Na altura 0, temos modulos unitários
Na altura 1, temos um grande módulo

\subsubsection{Características}

merges são finais.
algoritmo guloso, otimização local.
medidas monotonicamente relacionadas.
usa matriz de similaridade, que pode codificar qualquer critério.
complexidade alta
determinístico

%%%%%%%%%%%%%%%%%%%%%%%%%%%%%%%%%%%%%%%%%%%%%%%%%%%%%%%%%%%%
% 
% A técnica aqui apresentada 
% 
% Modelo de design OO: grafo
% 
% Clustering: coloca no mesmo grupo entidades similares
% 
% Como definir a similaridade entre entidades?
%   Parâmetros formais.
%   Parâmetros não-formais. (Anquetil)
% 
% Algoritmo HA. (Anquetil, Maqbool)
%   Explicação.
%   Similaridade entre clusters.
%   Critério de parada, ponto de corte.
% 
% Características do algoritmo HA.
%   merges são finais.
%   algoritmo guloso, otimização local.
%   medidas monotonicamente relacionadas.
%   usa matriz de similaridade, que pode codificar qualquer critério.
% 
% Técnicas.
%   k-NN: k-nearest neighbors.
%   SNN: shared nearest neighbors.
%   aplicar distância à matriz de similaridade para obter outra.
% 

%\section{Avaliação}




%Avaliação dos dados.
%  Forma.
%  Tamanho.
%  Densidade.
%  Presença de ruídos.
%  Coeficiente de clustering (tendência de clustering).

\section{Experimento} \label{sec:experimento}

Aplicamos o algoritmo hierárquico aglomerativo em 10 sistemas de diversos 
tamanhos, todos escritos em Java, conforme mostra a tabela \ref{tbl:sistemas}. 
Esse conjunto de estudo é um subconjunto dos 15 sistemas estudados por 
Bittencourt et al. no projeto Conformance.

\begin{table} \label{tbl:sistemas}[ht]
  \caption{Caraterísticas dos sistemas analisados: nome e quantidade de classes}
  \centering
  \begin{tabular}{l l r}
  \hline
  id & sistema & \# classes \\
  \hline 
  01\_junit       & JUnit               & 23  \\
  02\_villonanny  & VilloNanny          & 25  \\
  03\_easymock    & EasyMock            & 63  \\
  04\_pdfsam      & PDF Split and Merge & 68  \\
  05\_irc         & JWIRC               & 133 \\
  06\_SweetHome3D & Sweet Home 3D       & 97  \\
  07\_jvlt        & jVLT                & 235 \\
  08\_jedit       & jEdit               & 234 \\
  09\_robocode    & Robocode            & 251 \\
  10\_jgnash      & jGnash              & 321 \\
  \hline
  \end{tabular}
\end{table}

A métrica de similaridade entre classes usada no experimento se baseia nas
relações relações de dependência entre classes. 
Sejam $A$ e $B$ duas classes, e seja f($X$) o conjunto de classes das quais a 
classe $X$ depende. Definimos a similaridade entre duas classes, $A$ e $B$, 
como coeficiente de Jaccard entre f($A$) e f($B$):

\begin{equation}
\label{eq:jaccard}
\mathrm{sim}(A, B) = \frac{|\mathrm{f}(A) \, \cap \, \mathrm{f}(B)|}
                          {|\mathrm{f}(A) \, \cup \, \mathrm{f}(B)|}
\end{equation}

A métrica de similaridade entre grupos usada no experimento foi a
\emph{complete linkage}. Alguns trabalhos~\cite{Wu2005,Anquetil1999}
concluem que essa métrica produz resultados melhores do que a métrica 
\emph{single linkage}. Em especial, os melhores resultados foram obtidos
(em \cite{Wu2005}) com a altura de corte igual a 0.90, e por isso ela
foi utilizada neste experimento.

%Utilizamos 0.90 como altura de corte do dendograma, 
%
%Duas métricas de similaridade:
%- coeficiente de Jaccard aplicado sobre a matriz de adjacências da relação
%de dependência estática.
%- distância euclidiana aplicada sobre a matriz dos coeficientes de Jaccard.
%
%A intuição para a segunda métrica é que classes que são similares às
%mesmas classes devem ser similares entre si.
%
%Duas alturas de corte: (estudadas por [Anquetil?])
%- 0.75
%- 0.90

\subsection{Critérios de avaliação}
Para avaliar a qualidade das decomposições encontradas pelos algoritmos,
usamos dois critérios: semelhança com decomposição de referência e
tamanho dos clusters~\cite{Anquetil1999}.

O primeiro critério avalia o quanto a decomposição encontrada para um
sistema se assemelha à decomposição feita por um especialista no sistema.
Dada a dificuldade de se encontrar uma decomposição feita por um especialista,
consideramos a estrutura de pacotes do sistema (cada pacote representa um 
módulo).

Para comparar duas decomposições, utilizamos a métrica MoJo~\cite{Tzerpos1999}.
O MoJo entre 
duas decomposições é a quantidade de operações move --- mover uma classe
de um módulo para outro -- e join --- unir dois módulos --- necessárias
para se transformar uma decomposição em outra.

O segundo critério de avaliação é o tamanho dos módulos. Alguns algoritmos
tendem a formar muitos módulos pequenos ou então um grande módulo que engloba
quase todo o sistema, o que não é bom. 
Seguindo fulaninho, consideramos de bom tamanho os módulos que contêm
entre 5 e 100 classes (inclusive). A métrica NED (non-extreme distribution)
indica qual fração dos módulos do sistema tem um bom tamanho.

%%%%%%%%%%%%%%%%%%%%%%%%%%%%%%%%%%%%%%%%%%%%%%%%%%%%%%%%%%%%%%%%%%%%%%%%%%%%%%

\section{Resultados} \label{sec:resultados}

Desempenho: cerca de 10 minutos em um Atlhon 64 3000+ com 1 GB de RAM, Linux 2.6.18
A implementação desenvolvida para o experimento não foi otimizada.
(o último sistema demorou 2 minutos para cada métrica de similaridade)

A tabela ... sumariza os resultados obtidos no experimento. Como base de 
comparação, incluímos os resultados obtidos por Bittencourt et al. com o 
algoritmo k-means.

\subsection{Avaliação quantitativa}


Perguntas:
- Como euclidean afeta score, ned e mojo?
- Como altura afeta score, ned e mojo
--- Nos sistemas grandes

- Qual a melhor altura?
-- 90
- Qual a melhor similaridade?
-- jac

- Como o melhor algoritmo, com melhor altura e melhor similaridade, se compara 
ao k-means?

\subsubsection{Tamanho dos clusters}

\subsubsection{Semelhança com decomposição de referência}

\subsubsection{Score}


% \subsection{Avaliação qualitativa}

\subsection{Ameaças à validade}

Poucos sistemas
Uma linguagem
Apenas software livre (o processo de desenvolvimento afeta?)


%%%%%%%%%%%%%%%%%%%%%%%%%%%%%%%%%%%%%%%%%%%%%%%%%%%%%%%%%%%%%%%%%%%%%%

\section{Discussão} \label{sec:discussao}



%%%
%\input{metodologia.tex}

%\input{estudoDeCaso.tex}
%
%\input{conclusoes.tex}




\bibliographystyle{alpha}
\bibliography{evolucao2008}

\end{document}

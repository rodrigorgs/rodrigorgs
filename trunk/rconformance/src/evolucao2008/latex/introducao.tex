\section{Introdução} \label{intro}

As decisões mais significativas de um software formam a
arquitetura do software~\cite{Booch2006}. Essas decisões devem ser tomadas nos
estágios iniciais do processo de desenvolvimento, pois elas afetam um grande 
número de decisões ao longo do ciclo de vida do software.

Uma arquitetura é motivada pelos requisitos do software e pelas pessoas 
interessadas no sistema (os chamados \emph{stakeholders}).
É comum apresentar uma arquitetura em diversas visões, cada uma delas 
contemplando um subconjunto dos requisitos e dos 
\emph{stakeholders}~\cite{Kruchten1995}.
A visão lógica, por exemplo, é destinada principalmente aos desenvolvedores 
e trata de requisitos como a facilidade de manutenção e de compreensão do
software.

O campo de pesquisa denominado recuperação de arquitetura de software tem como
objetivo extrair aspectos da arquitetura de um sistema a partir de artefatos
criados durante seu desenvolvimento. A maioria dos trabalhos na
área se concentra na recuperação de uma visão lógica da arquitetura 
a partir da análise de código-fonte.

Muitos trabalhos empregam algoritmos de agrupamento
(\emph{clustering}) para decompor softwares em grupos elementos de design
(tipicamente funções e classes) relacionadas. Uma boa decomposição deve
refletir a visão de um especialista sobre o software.
%fornecer a desenvolvedores uma visão geral do sistema e deve revelar atributos
%de qualidade do software como reusabilidade, manutenibilidade e facilidade de
%compreensão.

Este trabalho avalia o algoritmo hierárquico aglomerativo através de sua
aplicação em dez softwares escritos em Java, uma linguagem orientada a
objetos. 
%
A seção \ref{sec:recovery} descreve de modo geral o processo de 
recuperação de arquitetura.
%
A seção \ref{sec:aha} apresenta o algoritmo hierárquico aglomerativo.
%
A seção \ref{sec:experimento} descreve o experimento realizado para avaliar
o algoritmo, e a seção \ref{sec:resultados} avalia os resultados desse
experimento.
%
Por fim, a seção \ref{sec:discussao} apresenta intuições sobre o uso de
algoritmos de agrupamento para recuperação de arquitetura.

%% trabalhos relacionados?? Anquetil e Lethbridge, Maqbool e Babri.



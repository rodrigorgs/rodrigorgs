\section{Introdução} \label{intro}

O conjunto de decisões mais significativas de um software formam 
arquitetura do software. Essas decisões devem ser tomadas cedo no processo
de desenvolvimento do software, pois elas afetam um grande número de decisões
que devem ser tomadas ao longo do desenvolvimento.

As decisões arquiteturais são motivadas pelos requisitos do sistema e dizem
respeito a uma parte de seus \emph{stakeholders}. É comum dividir a
arquitetura em diversas visões, cada uma delas contemplando um subconjunto
dos requisitos e dos \emph{stakeholders}. A visão lógica, por exemplo,
é destinada principalmente aos \emph{designers} e programadores e trata de
requisitos como a facilidade de manutenção e de compreensão do \emph{software}.
[cite 4+1]

O campo de pesquisa denominado recuperação de arquitetura de software tem como
objetivo extrair aspectos da arquitetura de um sistema a partir de artefatos
criados durante seu desenvolvimento. A maioria dos trabalhos na
área se concentra na recuperação de uma visão lógica da arquitetura de um
sistema a partir da análise de seu código-fonte.

Uma boa porção desses trabalhos emprega algoritmos de agrupamento
(\emph{clustering}) para identificar grupos de entidades de código-fonte
(tipicamente funções e classes). Um bom agrupamento deve
fornecer a desenvolvedores uma visão geral do sistema e deve revelar atributos
de qualidade do software como reusabilidade, manutenibilidade e facilidade de
compreensão.

%% trabalhos relacionados?? Anquetil e Lethbridge, Maqbool e Babri.



%Avaliação dos dados.
%  Forma.
%  Tamanho.
%  Densidade.
%  Presença de ruídos.
%  Coeficiente de clustering (tendência de clustering).

\section{Experimento}


\centering
\begin{tabular}{l l r}
\hline
id & nome & classes \\
\hline 
01\_junit       & JUnit               & 23  \\
02\_villonanny  & VilloNanny          & 25  \\
03\_easymock    & EasyMock            & 63  \\
04\_pdfsam      & PDF Split and Merge & 68  \\
05\_irc         & JWIRC               & 133 \\
06\_SweetHome3D & Sweet Home 3D       & 97  \\
07\_jvlt        & jVLT                & 235 \\
08\_jedit       & jEdit               & 234 \\
09\_robocode    & Robocode            & 251 \\
10\_jgnash      & jGnash              & 321 \\
\hline
\end{tabular}


\section{Resultados}

\subsection{Critérios de avaliação}
Para avaliar a qualidade das decomposições encontradas pelos algoritmos,
usamos dois critérios: semelhança com decomposição de referência e
tamanho dos clusters. [Anquetil]

O primeiro critério avalia o quanto a decomposição encontrada para um
sistema se assemelha à decomposição feita por um especialista no sistema.
Dada a dificuldade de se encontrar uma decomposição especialista, a
consideramos a estrutura de pacotes do sistema (cada pacote representa um 
módulo).

Para comparar duas decomposições, utilizamos a métrica MoJo [Cite].
O MoJo entre 
duas decomposições é a quantidade de operações move --- mover uma classe
de um módulo para outro -- e join --- unir dois módulos --- necessárias
para se transformar uma decomposição em outra.

O segundo critério de avaliação é o tamanho dos módulos. Alguns algoritmos
tendem a formar muitos módulos pequenos ou então um grande módulo que engloba
quase todo o sistema, o que não é bom. 
Seguindo fulaninho, consideramos de bom tamanho os módulos que contêm
entre 5 e 100 classes (inclusive). A métrica NED (non-extreme distribution)
indica qual fração dos módulos do sistema tem um bom tamanho.

\subsection{Avaliação quantitativa}

Desempenho: cerca de 1 hora em um Atlhon 64 3000+ com 1 GB de RAM, Linux 2.6.18


\subsection{Avaliação qualitativa}

\subsection{Ameaças à validade}




\section{Discussão}
